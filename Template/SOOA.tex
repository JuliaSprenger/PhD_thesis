\clearpage
\thispagestyle{empty}
\section*{Eidesstattliche Versicherung}
\label{sec:SOOA}

\vspace{2.5cm}

% Statement of original authorship - Needs to be in German
% see also here: https://www.wiso.uni-koeln.de/sites/fakultaet/dokumente/PA/formulare/eidesstattliche_erklaerung.pdf


% RWTH Version
Ich versichere hiermit an Eides Statt, dass ich die vorliegende Arbeit selbstständig  und  ohne  unzulässige  fremde  Hilfe (insbes.  akademisches  Ghostwriting) erbracht habe. Ich habe keine anderen als die angegebenen Quellen und Hilfsmittel benutzt. Für den Fall, dass die Arbeit zusätzlich auf einem Datenträger eingereicht wird, erkläre ich, dass die schriftliche und die elektronische Form vollständig übereinstimmen. Die Arbeit hat in gleicher oder ähnlicher Form noch keiner Prüfungsbehörde vorgelegen.

% Cologne Version
Hiermit versichere ich an Eides statt, dass ich die vorliegende Arbeit selbstständig und ohne die Benutzung anderer als der angegebenen Hilfsmittel angefertigt habe. Alle Stellen, die wörtlich oder sinngemäß aus veröffentlichten und nicht veröffentlichten Schriften entnommen wurden, sind als solche kenntlich gemacht. Die Arbeit ist in gleicher oder ähnlicher Form oder auszugsweise im Rahmen einer anderen Prüfung noch nicht vorgelegt worden. Ich versichere, dass die eingereichte elektronische Fassung der eingereichten Druckfassung vollständig entspricht.

\vspace{1cm}

\noindent
Die Strafbarkeit einer falschen eidesstattlichen Versicherung ist mir bekannt, namentlich die Strafandrohung gemäß § 156 StGB bis zu drei Jahren Freiheitsstrafe oder Geldstrafe bei vorsätzlicher Begehung der Tat bzw. gemäß § 161 Abs. 1 StGB bis zu einem Jahr Freiheitsstrafe oder Geldstrafe bei fahrlässiger Begehung.

\vspace{3cm}
\noindent
\textbf{\thesisauthor{}} 

\vspace{0.5cm}
\noindent
Köln, den xx.xx.20xx
