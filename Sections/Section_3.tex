\clearpage
\section{Making data usable - standardizing data representations}
\label{sec:Data}




\subsection{Neo}
Neo\footnote{Neo, \url{http://neuralensemble.org/neo}, RRID:SCR\_000634} \cite{neo09} is an open-source Python package for representing electrophysiology data in working memory. It offers interfaces for reading various electrophysiological proprietary and open file formats and represents the data in a generic way. Thus it forms the bases for a number of open software tools: The electrophysiology analysis toolkit\footnote{Elephant, \url{http://neuralensemble.org/elephant}, RRID:SCR\_003833} for analysis of spiking activity and local field potentials, OpenElectrophy\footnote{OpenElectrophy, \url{http://neuralensemble.org/OpenElectrophy}, RRID:SCR\_000819}, SpykeViewer\footnote{SpikeViewer, \url{https://spyke-viewer.readthedocs.io}} and Ephyviewer\footnote{Ephyviewer, \url{https://ephyviewer.readthedocs.io}} for visualization, Tridesclous\footnote{Tridesclous, \url{https://tridesclous.readthedocs.io}} for online and offline spike sorting, NeoAnalysis\footnote{NeoAnalysis, \url{https://github.com/neoanalysis/NeoAnalysis}} \cite{neoanalysis} for rudimentary visualization and analysis, NetworkUnit \footnote{NetworkUnit, \url{https://github.com/INM-6/NetworkUnit}, RRID:SCR\_016543} for validation testing of spiking networks.

The two main aspects of neo are 1) the interfacing to many different file formats, by providing reading capability for numerous proprietary formats and writing capability to selected open formats and 2) the standardized representation of electrophysiology data as a basis for further visualization and analysis steps. Using these aspects Neo can is typically used either as conversion tool from specialized to more generic formats or as runtime data representation for further processing.

\subsubsection{Features updates and current development}
The Neo version 0.3 was released in 2014 \cite{garcia_neo_2014}. Since then the software has been extended to be compatible with more data formats, the object model has been revised for better usability and the implementation has been improved for performance. In the following we describe the enhancements introduced between version 0.3 and version 0.7.

\paragraph{Interfaces to file formats}
Neo 0.7 is supporting additional file formats for reading, such as Axograph, OpenEphys, Stimfit, Kwik, Nix, Igor, Nest, Neuralynx, NSDF and BCI2000. The capabilities for reading the Axon, Blackrock, Brainvision, Brainware, Elphy, Intan Matlab structures, Neuroshare, Plexon, Spike2, Tdt, NeuroExplorer, Neuralynx, Igor, Elan, Micromed, RawMCS, WinWCP formats have been improved. Reading and writing capabilities have been improved for Nix and Pickle formats. PyNNText and PyNNNumpy formats are no longer supported. A new code design for readers has been implemented and the majority of readers has adjusted accordingly to enable improved loading performance and loading of subsets of data (RawIO implementation). 
\paragraph{Object structure and usability}
The code has been modularized for more flexibility and maintainability and a large number of unittests have been added. The object structure has been restructured for user friendliness and performance aspects by supporting sets of similar data entities in single objects instead of using individual data objects for each data entity (removal of dedicated array versions of data classes). A new relational container object `Channel\_Index` was introduced to simplify the representation of logical relations between data objects replacing `RecordingChannel` and RecordingChannelGroup` objects. Consistent deep copy functionality has been added for all data objects and additional internal consistency checks have been added. For the installation additional option were introduced, depending on the required file formats which need to be supported. The code style has been adjusted to follow the PEP8 guideline\footnote{Python Enhancement Proposal 8, \url{https://www.python.org/dev/peps/pep-0008}}. Support for Python 2.6 was dropped and consistent support for Python 3 was introduced.



% ...
% \subsection{Exemplary Figure}
% \label{subsec:Section_Name/fig}
% ...
% \begin{figure}[htbp]
%     \centering
%     \includegraphics[width=.5\linewidth]{./Figures/UoC_Logo.png}
%     \caption{Exemplary Figure}
%     \label{fig:UoC}
% \end{figure}
% 
% 
% \subsection{Exemplary Figure Referencing}
% \label{subsec:Section_Name/fig_rfs}
% 
% See Figure \ref{fig:UoC} for details. Additional information can be
% found in the footnote \footnote{Image taken from \url{https://en.wikipedia.org/wiki/File:Siegel_Uni-Koeln_(Grau).svg}.}.
