\clearpage
\section{Sharing data}
\label{sec:data}

Scientific progress is a sequence of scientific findings building on top of each other. These findings are classically hypothesis tested against experimental data and the resulting interpretation is communicated to other scientists via the publication of a manuscript. However, in recent years, the practice of publishing papers has been critizised for a lack of robustness. Attemps in a number of scientific fields, including among others life sciences, to reproduce published scientific findings failed to support the same conclusions \citep{Baker_2016, Fidler_2017, Pashler_2012, Ioannidis_2005, Goodman_2007, Ioannidis_2007, Collaboration_2015}. To qualitatively distinguish between different types of reproduction, a collection of terms has been applied to describe different levels of reproducibility of a study: reprodicibility, replicability, repeatability \citep{Plesser_2018, Drummond_2009}. However the specific interpretation of each of these terms highly depends on the scientific field they are applied in. For example repeatability in the sense of running the same experiment with identical components differs a lot in the computer sciences versus experimental sciences. Replicating a study by running the same code on the identical machine might be a realistic project, running the identical experiment in psycholoy is not \citep{Anderson_2016}. A widely accepted version of reproducibility terminology by \citet{Goodman_2007} is omitting 'replicability' and 'repeatability' for simplification and defines instead \citep[from][]{Plesser_2018}: 

\begin{description}
 \item[Methods reproducibility] \textit{'provide sufficient detail about procedures and data so that the same procedures could be exactly repeated.'}
\item[Results reproducibility] \textit{'obtain the same results from an independent study with procedures as closely matched to the original study as possible.'}
\item[Inferential reproducibility] \textit{'draw the same conclusions from either an independent replication of a study or a reanalysis of the original study.'}
\end{description}


This already introduces another important aspect of scientific publications: the underlying data. Reproducing findings of publications is harder for older publications \citep{Vines_2013}, which to a large part can be accounted for by the decreasing retrevability of the original datasets. The obligation to provide data to other scientists is a default introduced already decades ago for many funding initiatives (e.g. NIH since 2003\footnote{\url{https://grants.nih.gov/grants/policy/data_sharing/}}, BMBF recommends OpenAccess 2016\footnote{\url{https://www.bmbf.de/upload_filestore/pub/Open\_Access\_in\_Deutschland.pdf}}) for publications in a number of journals \footnote{\url{https://www.springernature.com/gp/authors/research-data-policy/data-availability-statements/12330880}}, however the readiness to comply is rather low (1/10 in \citep{Savage_2009}). To address the problem of reproducibility for original research articles, a number of journals were opened, which invite to publish replications of the original articles in computational science \footnote{ReScience, \url{http://rescience.github.io/}}, economics\footnote{\url{http://www.economics-ejournal.org/special-areas/replications-1}}, psychology\footnote{\url{https://www.apa.org/pubs/journals/xge/replication-articles}} and neuroscience \citep{Yeung_2017}.
To improve the situation of data availability, the Fair Principles were defined by \citep{Wilkinson_2016}   and a number of sites offering to host scientific data have become available: Zenodo\footnote{Zenodo, \url{https://zenodo.org/}}, Pangaea\footnote{Pangaea, \url{https://www.pangaea.de}}, BMC Research Notes\footnote{BMC Research Notes, \url{https://bmcresnotes.biomedcentral.com/} \& \url{https://bmcresnotes.biomedcentral.com/about/introducing-data-notes}}, DataDryad\footnote{Datadryad, \url{https://datadryad.org/}}, GIN\footnote{GIN, \url{https://gin.g-node.org/}}, Research Data Australia\footnote{RDA, \url{https://www.ands.org.au/online-services/research-data-australia/rda-registry}}, \citep{Assante_2016} and publishing data descriptor papers ScientificData\footnote{ScientificData, \url{https://www.nature.com/sdata/}}, DataScience\footnote{DataScience, \url{https://datascience.codata.org/}} \citep{Candela_2015}.
It has been shown, that mandated data archiving upon publication highly improves data availablility \citep{Vines_2013}. Going one step further, \citet{Chen_2019} show that making only the data available is not sufficient for reproducible science, data need to be accompanied by software packages used for further analysis. In the optimal case this encompasses detailed analysis motivation a complete analysis workflow. Making the effort of publishing the data is possible for large scale datasets, e.g. from particle physics \citep{Jomhari_2017}.

In this section, we provide an example of two complex published datasets from the neuroscientific field \citep{Brochier_2018}, demonstrating the size of data and metadata involved in such an experiment and the workflow used for data preparation.


% replication, reproducibility - different field different meaning
% 
% Scientific data as basis for findings
% Current system: experiment - analysis - paper -> manuscript publication
% Reprodicibility crisis in science
% 
% Different levels of reproducibility: reusablility, replicability, reproducibility... and the interpretation depends on the scientific field
% 
% first problem: data sharing, second problem: code and workflow sharing
% reasons for this: 'ownership of data, human data' additional effort of preparation not included in funding
% 
% immediate approach for past research: reproduction papers: ReScience,...
% solultions for 1): fair principles, data publications (add list of journals, repositories for hosting)
% solutions for 2) partially also addressed by data journals [open data is not enough]


% \todo{replication, reproducibility - different field different meaning https://www.nature.com/articles/s41567-018-0342-2}
% \todo{fair principles}
% 
% Reproducibility of scientific results is a recently highly debated topic.
% Making research data available for the scientific community is 
% 
% \todo{ReScience}
% 
% \todo{Data hosting sites: https://www.pangaea.de/submit/ https://bmcresnotes.biomedcentral.com/about/introducing-data-notes}
% \todo{Horizon 2020 - Open Access as a standard way of publishing  - text as well as research data %https://www.bmbf.de/upload_filestore/pub/Open_Access_in_Deutschland.pdf
% }
% 
% \todo{The Availability of Research Data Declines Rapidly with Article Age https://www.cell.com/current-biology/fulltext/S0960-9822(13)01400-0}
% 
% \todo{open data is not enough - include analysis and metadata -  CERN Analysis Preservation and REANA https://www.nature.com/articles/s41567-018-0342-2}
% \todo{Reuse of large data possible - CMS example (The CERN Open Data port) http://opendata.cern.ch/record/5500}


\todo{Abstract We publish two electrophysiological datasets recorded in motor cortex of two macaque monkeys during an instructed delayed reach-to-grasp task, using chronically implanted 10-by-10 Utah electrode arrays. We provide a) raw neural signals (sampled at 30kHz), b) time stamps and spike waveforms of offline sorted single and multi units (93/156 SUA and 49/19 MUA for monkey L and N, respectively), c) trial events and the monkey's behavior, and d) extensive metadata hierarchically structured via the odML metadata framework (including quality assessment post-processing steps, such as trial rejections). The dataset of one monkey contains a simultaneously saved record of the local field potential (LFP) sampled at 1kHz. To load the datasets in Python, we provide code based on the Neo data framework that produces a data structure which is annotated with relevant metadata. We complement this loading routine with an example code demonstrating how to access the data objects (e.g., raw signals) contained in such structures. For Matlab users, we provide the annotated data structures as mat-files.
% 
% Background and Summary 
% 
% We publish high-dimensional and multi-scale datasets that contain recordings from motor cortex with a 10-by-10 Utah electrode array during controlled reach-to-grasp movements for two monkeys (L and N). In particular, we provide the activities of a large number of simultaneously recorded single neurons (93 and 156, for L and N respectively) along with the continuous neuronal “raw” signals (sampled at 30kHz, and broadly band-pass filtered to 0.3Hz - 7.5kHz). To study the local field potential (LFP) [Mitzdorf85, Logothetis04, Einevoll13], a down-sampled and filtered version of the latter is provided for monkey N and can be computed from the data of monkey L. These high-dimensional parallel datasets provide the opportunity for neuroscientists and computational neuroscientists to study interactions in the cortical network during different epochs of a well described behavior. 
% 
% To date not many tools enable to study network coordination in high-dimensional data, since non-experimentalists, such as statisticians or theoretical/computational neuroscientists often do not have access to such data. Nevertheless, methods for correlation analysis of high-dimensional data that do not run into a combinatorial explosion and do have acceptable computing times are highly needed. This implies also the strong need for methods and tools that perform dimensionality reduction to reduce the complexity of the data and the computational load (e.g. [#Cowley13]). The development of such analysis methods requires to know the typical features of experimental data, such as non-stationarity in time and across trials, as well as deviations from typical theoretical assumptions (e.g. Poisson), in order to make the methods applicable to experimental data. If these features are ignored and not considered in the method, there is a considerable danger of generating false positive outcomes and potentially wrong interpretations of the results [#Gruen09_1126, #Louis10_127]. The data we publish allow non-experimentalists to get insights in features of cortical data from awake behaving animals.
% 
% Our data also provide the possibility to test and validate spike sorting methods (see [#Denker2014]). Besides the raw data we publish the corresponding spike data resulting from an offline spike sorting using the Plexon Offline Sorter. Other sorting methods can be applied to the same (raw) data, and differences in the results can be compared and analyzed.
% 
% Complex datasets, as the two provided here (high-dimensional, multi-scale during complex behavior), are a challenge for performing reproducible analysis. Besides the often rather variable nature of the circumstances under which such data were recorded, the data additionally experience a number of often interactively performed preprocessing steps before they can be used in actual data analyses. Without a detailed knowledge about all these steps, the actual data analysis may be biased or strongly affected. In most cases, electrophysiological data available as open source are in this respect not sufficiently annotated and documented. For this reason, we provide here a comprehensive description of how and under which circumstances the datasets were recorded as well as a detailed description of preprocessing steps that need to be considered before performing analyses on the data. We additionally publish a machine-readable format of these metadata including our parameters and results of the described preprocessing steps. We are aware that all this information may not be sufficient for the reproducible analysis of such data. The reason is that reproducible workflows including the provenance trail are not yet established for electrophysiological neuroscience, especially not for such complex experiments as presented here. Thus, we hope to make researchers from other fields, such as computer science, process engineering or others become interested to support the neuroscience field with solutions for reproducible research. Approaches from neuroinformatics / computer science are not yet sufficient to generate a complete provenance track of the processes involved (see e.g. [#Badia2015]). We provide here a concrete use case for such a development. 
% 
% In summary, we publish two datasets (one for each monkey, both having performed the same behavioral task) containing the raw neuronal data, offline sorted single and multi unit activity, the behavioral data, and metadata containing all information about each dataset including our parameters and results of the described preprocessing steps (see [#Zehl2016] on how to create a comprehensive metadata collection in a common file format). These are the first published datasets of massively parallel recordings from monkeys while they perform a complex instructed delayed reach-to-grasp task. [693/ 700 words]
% 
% Methods 
% 
% Subjects
% 
% All animal procedures were approved by the local ethical committee (C2EA 71; authorization A1/10/12) and conformed to the European and French government regulations. Monkey L and N are both Rhesus macaque monkeys (Macaca mulatta) which were trained to perform an instructed delay reach-to-grasp task for food reward (drops of apple sauce). Generally, the training started with getting the monkey accustomed to the experimenter and the monkey chair. After this setting-in period, the monkey had to learn a complex reach-to-grasp task in which he/she had to control both the grip type used to grasp an object and the force required to displace it. The training of the monkey was completed after he/she was able to perform the correct grip in 85% of the trials on average.
% 
% Monkey L 
% 
% Monkey L is female and was born on March 15, 2004. She started training in a preliminary version of the task in 2008. After a long break in 2009, she restarted training in June 2010 on the final version of the task. She had one phalanx missing on her right thumb and was therefore trained to perform the task with the left hand. The training was easy and fast with monkey L. She habituated rapidly to the training chair and could start the task training within a month from chair training onset. Generally, monkey L was described as being eager to work, quick and efficient during the task, but rather nervous. The surgery for the Utah-array implant took place on September 15, 2010. About 2 weeks after the surgery recordings started. At that time she had a body weight of 5 kg. Neuronal data were recorded during task performance until May 6, 2011. After this date, the quality of the recordings deteriorated suddenly and did not recover. On June 23, during a short surgery, the array wire bundle was cut, the connector was removed, but the array left in place. Monkey L is still alive. File names from monkey L are identified by the letter “l”.
% 
% Monkey N 
% 
% Monkey N is male and was born on May 15, 2008. He was initially trained to perform the task with the right hand between April 2012 and April 2013. After a first and disappointing recording period from the left hemisphere which ended in September 2013, he was retrained to perform the task with the left hand and implanted with another Utah array in the right hemisphere on May 22, 2014. In general, he was calmer than monkey L, but overall less motivated. He learned the task at a much lower pace, performed on average less trials per recording and was often less attentive than monkey L. When the recording period started he had a body weight of 7 kg. Neuronal data were recorded until January, 2015. As for monkey L, the quality of the recordings decayed gradually throughout the recording period. At the end of February, 2015, the monkey damaged the wire bundle between the connector and the array. As for monkey L, the wire bundle was then cut, the connector was removed but the array was left in place. Monkey N is also still alive. File names from monkey N are identified by the letter “i” (second letter of monkey N’s name, as reference for the recordings obtained from the second hemisphere). 
% 
% Surgery and Array Location
% 
% \centering  Implant locations of the Utah arrays. The figure displays the anatomical location of the Utah array of both monkeys after implantation as well as the fabrication settings of each array provided by Blackrock. a) Schematic drawing of a macaque cortex with implant location of array of both monkeys. Both arrays were implanted along the central sulcus and overlapping the putative border (dotted line) between primary motor cortex (M1) and dorsal or ventral premotor cortex (PMd or PMv) of the right hemisphere. b, d) Exact location of the array for each monkey in the close-up picture of the implantation site taken during the surgery (length of an array side is 4mm). The central sulcus, the arcuate sulcus and the superior precentral dimple are emphasized as thick black lines (to the left, right and top, respectively). c, e) Display the scheme of each array setting in a default array orientation where the wire bundle (indicated with white triangles in b - e) to the connector points to the right. Each array scheme shows the 10-by-10 electrode grid with the electrode identification numbers (IDs) defined by Blackrock (black numbers) and the location of the non-active electrodes (indicated in black as ID =-1). Gray numbers show an alternative set of connector-aligned electrode IDs, assigned based on electrode location with respect to the connector of the array, which are more practical for data analysis and comparisons across monkeys. In order to best cover the arm/hand representation of the primary motor cortex, each array was rotated for the implantation. The center of rotation is indicated by a colored triangle (b - e), stating below (in c and e) the degree of rotation for each array.
% 
% At the end of the training period, each monkey was chronically implanted with a Utah array (Blackrock Microsystems, Salt Lake City, UT, USA) in the motor cortex contralateral to the working hand. The array consisted of one 10-by-10 electrode grid with 96 active iridium oxide electrodes. Each electrode was 1.5mm long with an inter-electrode distance of 400\mum. The electrodes had on average an industrial impedance of 50k\Omega. The industrial impedance for each electrode is provided in the respective metadata file for each dataset. The electrodes were connected, along with one ground and two references, through a wire bundle of 4 cm length to a high-density CerePort Connector. 
% 
% The surgery was performed under deep general anesthesia using full aseptic procedures. Anesthesia was induced with 10 mg/kg intramuscular ketamine and maintained with 2-2.5 % isoflurane in 40:60 O2 air. To prevent cortical swelling, 2 ml/kg of intravenous mannitol infusion was slowly injected over a period of 10 min. A large craniotomy (20x20 mm and 30x20 mm for monkey L and N, respectively) was performed over the motor cortex and the dura was incised and reflected. The array was inserted using a pneumatic inserter (Array Inserter, Blackrock Microsystems) and covered with a sheet of an artificial non-absorbable dura (Preclude, Gore-tex) to avoid attachment of the array to the dura. The real dura was sutured back and also covered with a piece of an artificial absorbable dura (Seamdura, Codman). The bone flap was put back at its original position and attached to the skull by means of a 4mm x 40 mm strip of titanium (Bioplate, Codman). The array connector was fixed to the skull on the opposite side with titanium bone screws (Bioplate, Codman). The skin was sutured back over the bone flap and around the connector. The monkey received a full course of antibiotics and analgesics before returning to the home cage ([#Riehle2013]).
% 
% In both monkeys, the array was implanted a few millimeters anterior to the central sulcus with the wire bundle pointing in medio-caudal direction. We aimed at implanting the array in the arm/hand representation of the primary motor cortex with the most anterior electrodes encroaching upon the premotor cortex. For this reason, the arrays were rotated by 216 and 239 degrees for monkey L and N, respectively (cf. [fig:array_placements] b - e). The center of rotation was set to the bottom left corner of the default array orientation where the wire bundle to the connector points to the right (cf. [fig:array_placements] c and e). With respect to the precentral dimple and the spur of the arcuate sulcus, the array was located a few millimeters more lateral in monkey N than in monkey L (see [fig:array_placements]). Therefore, with respect to putative borders between MI, PMd, and PMv, the anterior electrodes of the array are assumed to cover part of PMd in monkey L and part of PMv in monkey N. 
% 
% Daily Routines
% 
% Recordings were performed on regular workdays. Weekends as well as holidays were usually excluded. During workdays the monkeys had unrestrained water access, but were only fed with dry food in the home cage. During the weekend and during holidays the food was supplemented with fruits and vegetables. A typical recording day consisted in: a) taking the monkey out of the cage, b) placing him/her in front of the experimental apparatus (see [subsec:Experimental-Apparatus]) in the primate chair, c) conducting several recordings of the neuronal activity while the monkey performed the behavioral task, and then d) returning the monkey back to its cage. A single recording session lasted between 10 to 20 minutes. The number of recording sessions per day depended on the motivation of the monkey, but on average a recording day lasted for 1.5 hour.
% 
% Task 
% 
% During a trial, the monkey had to grasp the object using either a side grip (SG) or a precision grip (PG). The PG had to be performed by placing the tips of index and thumb in a groove on the upper and lower sides of a cubic object, respectively (see [Fig:Control-of-Behavior-1] a, right). For SG, the tip of the thumb and the lateral surface of the other fingers were placed on the right and left sides of the object (see [Fig:Control-of-Behavior-1] a, middle). The monkey had to pull the object towards him/her against one of 2 possible loads requiring either a high or low pulling force (HF and LF, respectively). As a result, from the possible combinations of grip types and object loads, the monkey had to perform in total 4 different trial types (SG-LF, SG-HF, PG-LF, PG-HF). In each trial, the grip and force instructions for the requested trial type were provided to the monkeys independently through two consecutive visual cues (CUE and GO) which were separated by a one second delay. Both cues were coded by the illumination of specific combinations of 2 LEDs of a 5 LED cue panel positioned above the target object. Details on how the task, the trial scheme and the corresponding behavior of the monkey were controlled are stated in [subsec:Recording-and-Control].
% 
% Setup
% 
% 
% 
% The setup was organized in three major parts, the neural recording platform, the experimental apparatus, and the behavioral control system. The components of each part and their connections are summarized in [Fig:Setup_Overview] and described in more detail in the following sub chapters. Even though overall the same setup was used for both monkeys, the following differences are important to keep in mind: 
% 
% • The electrode configurations of the array are not identical between the two monkeys (cf. [fig:array_placements]).
% 
% • As headstage, Samtec or Patient cable was used for monkey L and N, respectively (cf. in [Fig:Setup_Overview], and [Fig:RecordingNeuralSIgnals]).
% 
% • The software version of Central Suite was updated between the recording periods of the monkeys. This led to differences in the data formats of the neural raw data (ns5 in monkey L vs. ns6 in monkey N) and in the content of the parallel saved data of the ns2-file which either only contained the behavioral signals from the target object manipulation (monkey L) or also included a downsampled and filtered version of the neural data (monkey N).
% 
% • The settings of the waveform window for the online spike shape extraction in Central Suite are not identical between the monkeys (cf. in [Fig:RecordingNeuralSIgnals]).
% 
% • The LabView implemented program to control and monitor the task and behavior was updated between the monkeys. This led to differences in the binary codings of the digital events stored in the nev file (cf. in [Table:Meaning-8bit-Combinations]).
% 
% The differences are described in more detail below and, in the referenced figures, marked in yellow and red for monkey L and N, respectively. 
% 
% 
% 
% Experimental apparatus
% 
% The experimental apparatus was composed by a target object, a table switch, a visual cue, and a reward system. On each recording day, the monkey was seated in a custom-made primate chair and placed in front of that apparatus. The non-working arm of the monkey was loosely restrained in a semi-flexed position. To control the home position of the working hand between the reach-to-grasp movements, the table switch which was installed close to the monkey at waist level, 5cm lateral to the mid-line, needed to be pressed down. The target object was a stainless steel rectangular cuboid (40mm x 16mm x 10mm) rotated 45 degrees around the horizontal axisTODO: this was vertical before. Please clarify. and pointing towards the monkey ([Fig:Control-of-Behavior-1]a). It was located 13cm away from the table switch at 14cm height. The posterior end of the object was attached through a low-friction horizontal shuttle to a counterweight hidden inside the apparatus, which was used to set the object load. The object load was set to one of two possible values to define the force type (LF and HF) needed for pulling the object in each trial by deactivating and activating an electromagnetic weight resting below the counterweight inside the apparatus. When activated, it attached to the counterweight and increased overall weight from usually 100\:g to 200\:g, which corresponds roughly to a pulling force of 1\:N and 2\:N for LF and HF, respectively. 
% 
% As already mentioned, the object was equipped with six sensors which monitored the monkey's reach-to-grasp behavior. Four force sensitive resistance sensors (FSR sensors) on the object surface provided continuous measurement of the grip forces applied on the object sides by the index and middle finger, as well as the thumb. The different activation patterns of these four FSR sensors, in particular the different placement of the thumb (see [Fig:Control-of-Behavior-1] a), were used to detect online if the correct grip type was performed. An additional FSR sensor was installed between the object and its counterweight. This FSR sensor was used to measure the horizontally applied force needed to oppose the corresponding object load. Due to the low, but still existing friction of the object moving inside the horizontal shuttle, the measured force signal of this sensor is not perfectly proportional to the horizontal force needed to lift the opposed object load, but sufficient to distinguish between LF and HF settings (cf., example in bottom right panel of [fig:Lilou-Fig1] and [fig:Nikos-Fig1]). The horizontal displacement of the object over a maximal distance of 15mm was measured by a hall-effect (HE) sensor. All sensors of the object are summarized in [tab:Behavioral-channels]. The visual cue system, composed of a square of five LEDs (size 10 x 10 mm), was located just above the target object and used to instruct the monkey about the requested behavior. While the central yellow LED was used to warn the monkey that a trial had started, the four red corner LEDs were used to code separately the grip and the force type for the requested trial type of each trial. In this context the illumination of the two left, the two right, the two bottom, or the two top LEDs coded for SG, PG, LF, or HF, respectively (see [Fig:Control-of-Behavior-1] b for illustration). The reward system consisted of a bucket filled with apple sauce and equipped with a feeding tube and a pump allowing to deliver on demand the reward (few drops of the apple sauce) to the monkey ([Fig:Control-of-Behavior-1] a).
% 
% Behavioral control system
% 
% The core of the behavioral control system is a custom-made Virtual Instrument (VI) in LabView that controls the digital event sequence and the requested behavior of each trial in a recording. A digital event reflects hereby the activation or deactivation of a physical device of the experimental apparatus. In this context, the LabView VI is responsible to activate and deactivate the LEDs of the visual cue system, the reward pump, and the electromagnet. The latter is not controlled by a digital event, but by an analog square signal that switches the magnet on or off. To control the requested behavior, the LabView VI monitors the monkey's manipulation of the table switch and the target object. The table switch as well as all sensors of the target object produce continuous analog signals that are digitized by the NI converter card and fed into the LabView VI of the setup computer (see [Fig:Setup_Overview] computer 2). The square signal of the table switch is then online reinterpreted as digital activation or deactivation event. [Fig:Control-of-Behavior-1] c displays a schematic diagram on how the physical devices of the experimental apparatus are connected to the setup computer and controlled and monitored by the LabView VI. We will now describe a typical execution of the LabView VI during a recording session in more detail. 
% 
% The possible trial types were set to SG-LF, SG-HF, PG-LF, and PG-HF, alternating with equal probability randomly in sequence between trials. Once the settings of the overall task were defined, the LabView VI was started to repetitively run and control the event sequence and behavior for each trial during the recording session. 
% 
% Each single trial was run and controlled as follows: 
% 
% The LabView VI only started a trial when the monkey deactivated the table switch by pressing and holding it down (home position, [Fig:Control-of-Behavior-1] a, left). This required not much muscle activity, but simply the weight of the monkey's hand on top of the smooth-running switch. If the table switch was deactivated, the LabView VI internally initiated a trial with a short time delay (TS-ON). In parallel, the program picked randomly one of the possible trial types (e.g., SG-HF) and activated or deactivated the electromagnet accordingly to fit the chosen load force of the object (e.g., activated for HF). To inform (or warn) the monkey that a new trial has started, the central LED was illuminated 400ms after the trial was initiated by the program (WS-ON). Four hundred ms after WS-ON the grip type was revealed to the monkey by illuminating the corresponding corner LEDs of the chosen trial type (CUE-ON, e.g., left LEDs for SG-ON). The LEDs of this first cue were turned off again after 300ms (CUE-OFF). The CUE-OFF was followed by a 1000ms preparatory delay at the end of which the monkey was informed about the upcoming force type by again illuminating the corresponding corner LEDs of the chosen trial type (GO-ON, e.g., top LEDs for HF-ON). This second cue also served as a GO signal for the monkey to initiate the movement which was registered by the activation of the table switch (SR-ON) when the monkey released it after a variable reaction time (RT). The execution of the movement was composed of reaching, grasping, pulling and holding the object in the position window for 500ms. The LabView VI controlled the movement execution online by checking the used grip type, the object displacement and the hold time. For checking the grip type, the grasp of the object was registered by small deflections of the FSR surface sensor signals caused by the monkey's fingers. A FSR sensor was registered as activated if the deflection surpassed a predefined threshold. The pattern of activated FSR sensors was then used by the LabView VI to control if the monkey performed the requested grip type. This meant, in particular, to check for SG and PG, if the FSR sensor on the right (GF side2), or on the bottom (GF pr2) of the object was activated by the monkey's thumb, respectively (see [Fig:Control-of-Behavior-1] a, middle and right). The other 2 sensors that measured force from the index and middle fingers for the 2 grip types (GF side1, and GF pr1) were not controlled online. If the correct grip was detected, the grip cue was illuminated again as a positive feedback. To check the object displacement, the LabView VI measured if the deflection of the HE sensor signal of the object was within the two defined position thresholds (4 and 14mm). The time point at which the displacement signal surpassed the lower threshold was used by the LabView VI to define the estimated start of the holding period (HS) online. If the object remained within the position window for 500ms after the HS was set, LabView activated the reward pump which provided the monkey with a drop of apple sauce as reward for a successful trial. The time until the reward pump was deactivated again by LabView was proportional to the duration of the object hold in the position window, with a maximum duration and with this a maximum amount of reward for a 500ms holding period. With this mechanism, both monkeys rapidly learned to hold the object at least 500ms in nearly all trials. In parallel to the deactivation of the reward pump, LabView turned off all LEDs to indicate that the running trial ended (WS-OFF). The monkey was allowed to release the object at its own pace as soon as it received the reward. A new trial sequence was started by LabView (TS-ON) as soon as the monkey returned to the home position (new deactivation of the table switch).
% 
% An abort of the described trial sequence by LabView (error trial) was triggered by the following three scenarios: (i) the monkey released the table switch before the GO cue, (ii) the wrong grip type was registered, and (iii) the object was not pulled and held long enough in the position window. In case one of these scenarios were registered by LabView the trial was aborted. For monkey L, the LabView VI provided additionally a negative feedback when aborting a trial by flickering all LEDs three times. 
% 
% As displayed in [Fig:Control-of-Behavior-1] c. the behavioral control system was connected to the NSP of the Cerebus DAQ system to store the trial event sequence and the monkey's behavior of each trial in a recording along with the neural data registered by the neural recording platform. For this, the analog signals of the sensors of the target object were copied from the NI connector block to the analog input port of the Cerebus System NSP via DC coupled BNC cables and connectors. In the NSP they were digitized with a 16-bit resolution at 0.15 mV/bit and a sampling rate of 1kHz and saved under the channels ids listed in [tab:Behavioral-channels] in the ns2-file (see [sec:data_rec]). All digital or digitized events that register the activation and deactivation of the table switch, the LEDs of the cue system, and the reward pump, as well as the internally generated digital trial start event (TS-ON) were coded as a 8-bit binary signal (see [Table-digital-signals-1]) and transferred via the NI connector block to a 16-bit DB-37 input port of the NSP where they occupy the first 8 digits (remaining digits are set to 1). In the NSP the now 16-bit binary signal of each event was stored in its decimal representation and with its corresponding time point in the nev file (see [Table-digital-signals-1] and [Fig:RecordingNeuralSIgnals]).
% 
% \centering 
% 
% Overview of the six object sensors used to monitor and control the monkey's behavior. The first four force sensitive resistance (FSR) sensors are used to monitor the applied grip type. They are located on the surface of each object side and are activated by the touch of the corresponding monkey's finger. The fifth FSR is located at the spring counterbalancing the pull resistance of the object and is used to measure the pulling force applied by the monkey. The hall-effect sensor (HE) is located along the low-friction shuttle of the object and used to measure the position of the object. The signals of all sensors are saved in the ns2 with the stated channel ID and label (cf. [Fig:RecordingNeuralSIgnals]).
% 
% Neural recording platform
% 
% 
% 
% The recording of the neural signals was performed using a neural recording platform with components produced by Blackrock Microsystems (Salt Lake City, UT, USA, www.blackrockmicro.com). The platform consisted of the multi-electrode Utah array, a headstage, and a Cerebus data acquisition (DAQ) system. The latter is composed of a Front-End Amplifier, a real-time Neural Signal Processor (NSP) and the control software, Central Suite (version 4.15.0 and 6.03.01 for L and N, respectively), running on Windows XP for L, and Windows 7 for N on the setup computer 1 (see Fig. [Fig:RecordingNeuralSIgnals]). The Cerebus DAQ system was also connected to the behavioral control system via the NI connector block to save the analog behavioral data and digital trial event signals that were described in the previous section in parallel with the neural signals. All data were transmitted from the NSP via an ethernet cable to be saved first locally on the setup computer 1. After a recording day, all recordings were transferred to a data server. In the following, we will describe the function of the different components of the neural recording platform in more detail.
% 
% The implant location of the Utah array, as well as the electrode configuration of the array of each monkey was described previously (see [ssec:surgery_arrayloc] and [fig:array_placements]). The electrode identification numbers (IDs) are determined by how the electrodes of the array are wired and connected to the Cerebus Front-End Amplifier. See [subsec:ch_id_config] for details.
% 
% The analog Blackrock headstage with unity gain (Samtec for monkey L, and Patient Cable for monkey N) was used to reduce the environmental noise. Overall, the reduction of the noise was better with the Patient Cable than with the Samtec headstage.
% 
% In the Front-End Amplifier, each of the 96 neural signals was differentially amplified with respect to the reference input of its corresponding connector bank (gain 5000) and filtered with a 1st-order 0.3Hz high pass filter (full-bandwidth mode) and a 3rd-order 7.5kHz Butterworth low pass filter. After that, the band-pass filtered neuronal signals were digitized with a 16-bit resolution at 0.25V/bit and a sampling rate of 30kHz, in the following called “raw signal”. The digitized signals were converted into a single multiplexed optical output and transmitted via a fiber-optic data link to the NSP. In the NSP the raw signals were saved in a ns5-file for monkey L and in a ns6-file for monkey N. The file format depended on the firmware and software version of the Cerebus DAQ system. In addition to the neural signals, the NSP received the analog behavioral signal recorded by the behavioral control system via the analog input port. These behavioral signals were digitized and saved with a sampling rate of 1kHz in a ns2-file. For monkey N, the ns2-file also contained a filtered and downsampled version of the raw signals, in the following called “LFP data”. To extract the LFP data, a copy of the raw data was online digitally low-pass filtered at 250Hz (Butterworth, 4th order), and downsampled to 1kHz within the NSP.
% 
% The NSP performed also an online spike waveform detection and classification controlled via the Central Suite software. The sorted spikes were used for a first online inspection of the data as well as for selecting and saving the spike waveforms for offline sorting. For this purpose the neuronal raw signals were for monkey L online high-pass filtered at 250 Hz (Butterworth, 4th order) and for monkey N band-pass filtered between 250Hz and 5kHz (Butterworth, 2nd order). Afterwards, the waveforms were detected by threshold crossing (manually set). These waveforms were then sorted by requesting the signal from identified neurons to follow through up to five hoops set by the user (all individually for each channel). To get an overview of the quality of the data during the recordings, the sorted waveforms were displayed in the online classification window provided by Central Suite.
% 
% The thresholds (one for each channel) for the spike waveform detection were not modified during a session and were saved in the nev-file for each session along with all other settings (e.g. filter setting etc) and configurations of Central Suite. The data and corresponding settings of Central Suite can also be inspected offline using the Blackrock software CentralPlay even in the absence of the Blackrock hardware system. Each time the high-pass filtered signal passed the threshold, a snippet of 1.6ms (48 samples) for monkey L and 1.3ms (38 samples) for monkey N was cut and saved as potential spike waveform. The snippet was cut with 10 sample points before threshold crossing and 38 or 28 points after for monkey L or N, respectively. Waveforms identified as potential single units (online sorted spikes) were labeled with IDs from 1 to 16. Unsorted waveforms were labeled with ID 0. These potential spike waveforms were saved together with their respective time stamps in the nev-file. Due to the high number of electrodes, online spike-sorting was moderately reliable. We therefore decided to re-sort spiking activity offline on each channel using the Plexon Offline Spike Sorter (Plexon Inc, Dallas, Texas, USA, version 3.3, for details see [subsec:Data-Preprocessing]). Results of offline sorting were saved in a copy of the original nev-file with an updated file name. 
% 
% All data files (nev, ns5/6, ccf) were saved on disk and backed-up on a data server at the end of the recording sessions. The information collected here are partly taken from [#Riehle2013, #Zehl2016].
% 
% 
% 
% 
% 
% Origin of the channel IDs
% 
% The neuronal signal inputs to the Front-End Amplifier were grouped into four banks (A-D or 0-3) from which only the first 3 were used. Each bank consists of a male header with 34 pins of which 32 were the neuronal signal input channels. The other two channels served as reference and ground, respectively. In Central Suite, the identification (ID) number of each electrode of the array is defined by the position on the input bank and pin of the Cerebus Front-End Amplifier. For this Central Suite multiplies the bank ID (0, 1, 2, or 3) with the number of pins for neural signal input channels (32) and adds the ID of the pin the electrode is connected to (cf. ID conversion in [Fig:RecordingNeuralSIgnals]). The electrode wiring of the Utah array is, though, not coordinated to the input banks of the Front-End Amplifer which leads to spatially unordered electrode IDs. Nevertheless, Utah arrays are fabricated usually in the same way where the corner electrodes are unconnected leading to a default (unordered) electrode ID configuration (cf. electrode configuration of monkey N in [fig:array_placements]). If in the fabrication process one of the corner electrodes was registered to be of significantly higher quality than any other electrodes of the grid, the corner electrode was connected instead and thereby changed the corresponding electrode configuration (cf. electrode configuration of monkey L in [fig:array_placements]). This led to the different ID sequences of the arrays for monkey L and N (see [fig:array_placements]). To facilitate the comparison of results between arrays with different electrode configurations, we assigned new IDs that reflect the spatial organization of the array. For this we used as reference the lower left corner electrode, when the connected wire bundle is showing to the right. These fabrication-independent, connector-aligned IDs increase linearly from bottom left to top right, line by line. They are also shown in [fig:array_placements]d as gray numbers in the array sketch, which thereby provides the mapping of the Blackrock IDs to the connector-aligned IDs. 
% 
% 
% 
% 
% 
% Data Preprocessing
% 
% After the recordings, a number of preprocessing steps (pre in the sense of before the actual upcoming data analysis, but being the post-processing after the recording) were performed as described below. This includes (i) the translation of the digital events from their binary codes set by the DAQ system to a human-readable format putting the events in context of the expected executed trial event sequence, (ii) the offline detection of behavioral trial events and object load force from the analog signals recorded by the sensors of the target object, and (iii) the offline spike sorting.
% 
% Translation of digital events to trial events 
% 
% Table [Table:Meaning-8bit-Combinations] lists the 8-bit combinations that were sent by LabView to the Experimental Apparatus to control the behavior. Following a binary to decimal conversion, they were saved as event codes (Table [Table:Meaning-8bit-Combinations]) during the experiment along with their time stamps in the .nev file. In the first preprocessing step, these event codes were translated to a human-readable format and put into context of an expected trial event sequence. The validation against the latter was used to identify incomplete, correct and error trials. Error trials were further differentiated into error types (e.g., grip error). This digital event translation and interpretation is performed automatically within the reach-to-grasp loading routine. 
% 
% Translation table of the 8 bits to the event codes and their behavioral meaning (labels). The 8 bits (see Table [Table-digital-signals-1] for their meaning) were sent from LabView to NSP during the trial sequence (Fig. [Fig:Control-of-Behavior-1]). The event codes are the decimal version of the bit sequence assuming another byte with all bits set to 1 in front. The event codes are found in the .nev files with a time stamp and indicate the occurrence of a stimulus / behavioral event as indicated in the center colum ('label'). Due to different versions of the LabView control program for monkey L and N (see text for details) the event codes for the same label may be different for the two monkeys. Also some event codes do not have a concrete meaning (miscellaneous) and occur sporadically in the .nev file due to a mistake in the sampling of the digital events - they have to be ignored. In the table the event codes are sorted in sequential order from top to bottom with respect to the task, i.e. their order corresponds to the sequence found in the .nev file in an successful trial. TB: Misc: They correspond to erroneous codes occuring occasionaly when the digital coding overlaps across 2 consecutive samples (time stamps) in the NEV files. They must be ignored when downloading the data as they have no real meaning (certainly, we need to add a comment in the manuscript about these). 
% 
% Preprocessing of behavioral analog signals 
% 
% Some behavioral events such as the monkey touching the object or the onset of the object displacement by the monkey were controlled during the experiment, but their online-detected timing was approximate and not saved (see details in section [subsec:Recording-and-Control]). However, these events can be relevant for data analysis and they were thus computed offline from the analog signals of the four FSR sensors measuring the monkey's grip and the HE sensor measuring the object displacement. We implemented a custom-made Matlab Event-Detection toolbox to detect 8 specific events: the precise timing of object touch (OT) and object release (OR) from the force traces as well as the timing of displacement onset (DO) and object back to baseline (OBB) from the displacement trace, and finally the onset and offset of the plateau phase in the force and displacement traces. The plateau phase of the displacement signal indicates the timing and stability of the holding period, and its onset is used to calculate offline the hold start (HS) signal. The toolbox performed an automatic detection of these events and their timing was first approximated by threshold crossing and then fine-tuned by back-comparison of the traces with baseline level from the point of threshold crossing. Since the automatic detection was prone to errors, the trials were visually inspected one by one and the timing of the automatically detected events were manually corrected if they did not match the event times as visually identified. In addition, a Matlab script was used to inspect the load force traces in each trial to control if the actual object load corresponded to the programmed object load. This procedure ensured that the electro-magnet controlling the object load was properly activated throughout the recording session. 
% 
% Offline spike sorting
% 
% The spike waveforms which were extracted and saved (in the nev file) during the recording were offline sorted using the Plexon Offline Sorter (version 3.3.3). To keep the variability in the half-manual spike sorting at a minimum, all sortings were performed by the same person (A. Riehle). The spike sorting started with loading the complete nev file of a session into the Plexon Offline Sorter. The spike sorting was performed on a duplicate of the data file to keep the original data intact. We started by joining all different waveforms extracted online from each channel separately back again into one pool and initially marked as “unsorted waveforms” in the Plexon Offline Sorter. Thereby, we ignored the result of the preliminary online waveform sorting (units 0-16 in the nev file) that was performed during the recording via Central Suite software, which served solely to extract waveforms and gain an overview of the quality of the spiking activity. For the invalidation of cross-channel artifacts (e.g., chewing artifacts) all waveforms that occurred simultaneously on a defined percentage of channels (70%) were marked as “invalidated waveforms” in Plexon Offline Sorter. Such artifacts occurred only in the recording session of monkey L. Furthermore, a waveform rejection was performed. Thereby all waveforms of abnormally large amplitude and/or atypical shape on a channel were manually marked as “invalidated waveforms” in Plexon Offline Sorter.
% 
% The actual spike sorting was then performed on the remaining unsorted waveforms (i.e., those not marked as invalidated waveforms) individually for each channel. We used different algorithms to split these waveforms into clusters in a 2- or 3-dimensional principal component (PC) space. The dimensionality of the PC space was chosen according to the best separation. The main algorithms used were K-Means(-Scan) and Valley Seeking (chosen according to the best separation). We used a fixed threshold for outliers (a parameter to be determined in the Plexon Offline Sorter) between 1.8 (K-Means) and 2 (Valley Seeking) to get comparable sorting results. The spikes of the sorted clusters were then controlled using the inter-spike interval (ISI) distributions and the auto- and cross-correlation plots. Units were ordered manually from best to worst (assigning increasing unit IDs 1-16 in the Plexon Offline Sorter) by considering the amplitude of the waveform (the higher the better), the outcomes of the ISI analysis (no or low number of spikes with an ISI smaller than 2 ms), the correlation histograms, and identifiable cluster shapes. Waveforms in the cluster with the highest unit ID (worst) on a given channel may contain multi-unit activity. Clusters with unacceptable outcomes (completely or partly overlapping waveforms), including those with only a few spikes, left assigned as “unsorted waveforms” in Plexon Offline Sorter. This offline spike sorted nev file was saved under the file name of the original nev file with an added two-digit numeric postfix (e.g. -01). In this file, unit ID 255 contains invalidated waveforms, unit ID 0 contains the unsorted waveforms (that may enter a further cluster analysis for spike sorting), and unit IDs 1-16 contain the time stamps and waveforms of the sorted single- or multi-units (as in the Plexon Offline Sorter). Unit IDs that are considered to represent multi-unit activity are documented in the metadata. The nev file with the sorted units can be loaded again into the Plexon Offline Sorter to visualize all the sorted spikes and rework the spike sorting.
% 
% Code availability
% 
% All available code required to access the data as described in [sec:usage] is stored along with the datasets. The provided code includes, in particular: (i) a snapshot of the Python Neo package, (ii) a snapshot of the Python odML package, (iii) the custom-written ReachGraspIO extending the Neo package, (iv) the example script shown and described in [sec:usage], (v) the code shown and described in [sec:usage] demonstrating how to access the data in Matlab.
% 
% In addition to these frozen versions of the code, we recommend to use updated versions of the code to benefit from future enhancements, bug fixes and increased compatibility with future Python releases or novel applications that rely on recent versions of Neo and/or odML. Complete link collections to the python-neo and python-odML libraries can be found at http://neuralensemble.org/neo/ and http://www.g-node.org/projects/odml, respectively. Importantly, both projects are hosted and version-controlled via Github at https://github.com/NeuralEnsemble/python-neo and https://github.com/G-Node/python-odml. Updated versions of the ReachGraspIO and all example code can be found on Github under https://github.com/INM-6/reachgrasp-dataset.
% 
% Data Records 
% 
% All data and metadata are publicly available via the data portal of the German Neuroinformatics Node (G-Node) of the International Neuroinformatics Coordination Facility (INCF), called GNData (http://g-node.github.io/g-node-portal/). [Table:datafiles_overview] provides an overview of the name, size, and content of all files for each published dataset of monkey L and N. The datasets of both monkeys consist of four parts: (i) the primary data are provided as the original data files obtained from the Central Suite software stored in the data format specified by the manufacturer (in particular, nev, ns5 and ns6 format) of the neural recording platform, Blackrock Microsystems; (ii) an offline sorted version of the neural spike data (cf. [subsec:Offline-spike-sorting]) is provided in a second nev file; (iii) metadata are provided as one file per dataset in the odML format [#Grewe2011, #Zehl2016]; and (iv) a mat file is provided containing the continuous neural raw data together with the offline sorted spike data, both annotated with the corresponding metadata.
% 
% 
% 
% Dataset information
% 
% Overview of recording days of the published datasets. For both monkeys, we chose to publish the first dataset (rec*-001) of the recording day. For details on the published datasets see [Table:datafiles_trials] and [Table:datafiles_overview].
% 
% The dataset l101210-001 from monkey L is the first out of 9 recording sessions conducted on Friday, December 10, 2010, while the dataset i140703-001 from monkey N is the first out of only 3 recording sessions conducted on Thursday, July 3, 2014. Both datasets were recorded in the late morning. The following recording day went on for nearly one hour and a half for monkey L, and one hour for monkey N. Although the recording from monkey N lasted with 16:43 min several minutes longer than the recording from monkey L with only 11:49 min, monkey L executed 204 trials, while monkey N only performed 160 trials in total. However, monkey L performed only ~70% of all trials correctly, whereas monkey N successfully completed ~90% of all trials during the recording (cf. [Table:datafiles_trials]). Nonetheless, the high percentage of error trials in monkey L are mainly caused by an too early movement onsets reflecting the eagerness, but also the nervousness of the monkey L's character (cf. [sssec:monkeyL]). In contrast to these error types, monkey L used only 12 times the wrong grip compared to monkey N who performed an incorrect grip type 16 times during the recording. 
% 
% Overview of trials performed during the published datasets. Of the stated number of error trials, the monkey L and N used the wrong grip type in 12 and 16 trials, respectively. In the remaining error trials the monkeys initiated the movement too early. Trial types were altered randomly in the recordings which led to slightly different trial numbers for the different trial types.
% 
% For both monkeys the trial types alternated randomly between trials leading to slightly different numbers of trials with the same trial type in the each dataset (cf. [Table:datafiles_trials]). 
% 
% 
% 
% The quality of the spiking activity in the datasets of both monkeys was high, which allowed us to perform a relatively robust offline spike sorting with high numbers of single unit activity (SUA) distributed over all electrodes of the array (for details see [tab:datafiles_unitactivity]). For details on how the offline sorting was performed and checked please have a look at [subsec:Offline-spike-sorting] and [sssec:tech_val_quality_spike]. 
% 
% Information on metadata framework
% 
%  All metadata information about the experiment, the subject, the setup, the settings, the processing of the data etc were originally distributed over several source files, but were collected offline in one metadata file per recording using the odML metadata framework. This odML framework was first introduced by [#Grewe2011] and is a supported metadata management project of the G-Node (http://www.g-node.org/projects/odml). odML files are machine readable and can be used via application programming interfaces (APIs) in Java (https://github.com/G-Node/odml-java-lib), Python (https://github.com/G-Node/python-odml) and Matlab (https://github.com/G-Node/matlab-odml). Moreover, odML files are human readable and can be screened best by either using the odML Editor which is part of the odML Python API, or using a browser via a metadata stylesheet available as download on the G-Node project website. For details on how to manage metadata for such a complex experiment using the odML framework please have a look at [#Zehl2016]. This reference also includes tutorial like code examples on how to use odML files in Python and Matlab (see main article as well as supplementary material).
% 
% Technical Validation
% 
% In addition to the above described preprocessing steps that needed to be performed to gain more content of the raw data, some technical validations of the data also had to be conducted. These technical validations include the correction of the irregular alignment data files of the Cerebus DAQ system and a general quality assessment of the data. In order to validate the quality of the recording, a series of algorithms were applied to the data. On the one hand the quality of the LFP signals was assessed per electrode and per trial by evaluating the variance of the corresponding signal in multiple frequency bands. On the other hand the quality of the offline sorted single units ([subsec:Offline-spike-sorting]) was determined by a signal-to-noise measure. In addition, noise artifacts occurring simultaneously in the recorded spiking activity were detected and marked. In the following, we explain these technical validation steps in detail.
% 
% Correction of data alignment
% 
% The ns6 file starts always 82 samples later than ns5, ns2 and nev files. This miss-alignment is caused by an error in the Blackrock recording software. However, this shift is correctly recorded in the ns6 file, and therefore will be automatically corrected in the generic Neo loading routine (cf., BlackrockIO in [sec:usage] below). In addition, due to the online filter procedure, the LFP signals in the ns2 file are delayed by approximately 3.6 ms with respect to the time stamps in the nev file and the analog signal of the ns6 file. This offset was heuristically determined, documented in the metadata file, and can be automatically corrected for by the experiment-specific loading routine (cf., ReachGraspIO in [sec:usage] below). Note that the time stamps of the spike times provided in the nev file correspond to start of the waveform and not to the time point of threshold crossing.
% 
% Quality assessment
% 
% The occurrence of noise in electrophysiological recordings is to a certain degree unavoidable and therefore needs to be carefully examined. It depends to a large extent on the quality of the headstage used to record the neurophysiological data. In our data, two different types of headstages were used for the two monkeys - the Samtec-CerePort headstage (monkey L) and the Patient Cable (monkey N). The former is much more sensitive to noise than the latter. The type of noise, its cause and appearance in the data is quite variable. Depending on the direct influence of the different types of noise on subsequent analysis methods, one needs to balance the corresponding data rejection between being very permissive and very conservative. For this reason, it is wise not remove or delete data of bad quality, but instead mark them with the judgment of a corresponding quality assessment procedure. For the here published datasets, we provide the results of our quality assessment of the electrodes, trials and spiking units along with the analysis parameters of the used procedure in the odML metadata files for each recording. The reach-to-grasp IO integrates this information by annotating the corresponding data objects in Neo. This approach not only allows the user to finally decide which data to reject for an analysis, but also provides the opportunity to provide different quality assessments of the same electrode, trial and unit at the same time. This is helpful if one considers that certain types of noise can differently contaminate signals in different frequency bands. For the here published datasets, the quality of the recorded signals was therefore separately tested for the sorted spike data and different frequency bands of the LFP data. The used corresponding procedures are described in detail below. 
% 
% LFP data quality 
% 
% The LFP data were examined for noise in three broad frequency bands excluding the 50Hz European line noise (low: 3Hz - 10Hz, middle: 12Hz - 40Hz, high: 60Hz - 250Hz) in each session individually. The goal of the quality assessment was, first, to detect channels with a noisy signal throughout the session and, second, to detect noisy trials in the remaining “clean” channels. To do so, the analog signals of each electrode were first z-scored and filtered in the three frequency bands (low, middle, and high) using a Butterworth filter (of order 2, 3, and 4, respectively). For each frequency band the quality assessment analysis was carried out separately. The detection of noisy electrodes was performed in three steps: 
% 
% step 1 The variance of the filtered analog signal of each electrode was calculated over the complete session. 
% 
% step 2 Out of the 96 resulting variance values, outliers were identified as those values outside a user-defined range. The range was defined as follows: (i) values between a lower (e.g., 25th) and an upper (e.g., 75th) percentile (L and U), (ii) the range of acceptable values was defined by [L-w\cdot(U-L),U+w\cdot(U-L)],where w is a user-defined whisker coefficient (e.g., w=3). 
% 
% step 3 The analog signals classified as outliers in step 2 were visually controlled by comparing them to the analog signal of an electrode with a typical variance value. If the results were either too conservative or too permissive, the detection procedure was repeated by manually adapting the chosen parameters (L, U, and w), correspondingly. 
% 
% The electrode IDs of the final outliers as well as the parameters chosen for their detection were saved in the odML metadata file of the corresponding recording and marked as noisy for the tested frequency band. 
% 
% For the remaining non-noisy electrodes, an analogous procedure was carried out afterwards to detect noisy trials. The procedure differed in one respect: the variance of the filtered analog signal was calculated for each trial on each electrode separately. At the end, the trial IDs of the identified outliers were pooled and marked as noisy for the tested frequency band on all electrodes. The marked trial IDs were saved in the odML metadata file of the corresponding recording together with the chosen analysis parameters for their detection. Note again that with this procedure a trial is marked as noisy on all electrodes as soon as it is classified as noisy on one electrode.
% 
% Spike data quality 
% 
% To test and judge the quality of the spike data, the results of the offline spike sorting were controlled first, for the signal-to-noise ratio (SNR) from the waveforms of the identified single units and second, for the occurrence of hyper-synchronous event artifacts.
% 
% 1. To calculate the SNR for each identified unit in the sorting results a method introduced by [#Hatsopoulos04_1165] was used. The SNR was defined as the amplitude (A, trough-to-peak) of the mean waveform (<w>) divided by twice the standard deviation of the waveform noise (SD_{noise}) of the defined unit (u): SNR_{u}=A_{<w>}/SD_{noise}\cdot2,where SD_{noise} was computed by averaging the standard deviations (SDs) obtained from each sample point across the original waveforms (SD of the waveform noise adapted from [#Nordhausen96_129, #Suner05_524]. For all identified single units in the datasets published here, the determined SNRs ranged between 1.5 and 12. Corresponding to [#Suner05_524] the quality of the spike sorting of an identified unit is good if the SNR is above 4, is fair if the SNR ranges between 2 and 4, and is poor if the SNR ranges between 1 and 2. Units with an SNR below 1 are not considered as signals. For a conservative analysis of the spike datasets, we recommend to use only single units with a SNR of 2.5 or higher, which was our choice in e.g. TODO: this needs to be replace by correct ref (the paper is out is it not?)[#Torre16_underrevision]. The results of the SNR analysis of the performed spike sorting were saved in the odML metadata file of the corresponding recording and units were annotated accordingly. 
% 
% 2. TODO: Check with Emiliano(s papers). Synchrofact detection based only on SUA and not on all spikes?Since correlation analysis of spike data is very sensitive to cross-electrode artifacts which would produce unwanted false positive results, we controlled the sorted spike data on their original time resolution (\delta=1/30ms) for potential occurrence of hyper-synchronous event artifacts. For this, we computed the population histogram, i.e. the sum of the spikes across all sorted single units in the dataset in bins of \delta=1/30ms (sampling resolution of the data), and detected if there were entries\ge2. To our surprise these hyper-synchronous spikes, which are likely to be attributed to cross-channel noise, survived the spike sorting including the cross-channel artifact removal by the Plexon Spike sorter. We indeed detected these spike artifacts during a preliminary analysis of a previous study (TODO: this needs to be replace by correct ref (the paper is out is it not?)[#Torre16_underrevision]). The number of single units participating in these events ranged from 2 to over 30 and a statistical analysis showed that the frequency of their occurrence largely exceeded the expected value considering the observed population firing rate. Furthermore, a \delta-binned time histogram of the population spiking activity triggered around the occurrence times of the hyper-synchronous events revealed also increased spiking activity in the preceding or following bin of the event. For a conservative analysis of the spike datasets, we recommend to treat the spikes participating in a hyper-synchronous event as well as the spikes occurring within a short time interval around this event ({\scriptstyle \pm\delta}) as artifacts of unknown origin and to remove them subsequently before performing any analysis of the spike data.
% 
% In [#Torre16_underrevision] we combined both quality assessments of the spike data and only considered spikes with a SNR>2.5 and additionally removed all hyper-synchronous events with \ge2 spikes. 
% 
% Usage Notes 
% 
% In the following, we describe how the provided data files ([sec:data_rec]) can be practically used in a data analysis scenario. To this end, we first briefly present the open source software libraries we recommend to use in order to access data and metadata using the Python programming language. We also demonstrate how to merge data and metadata in a common representation that facilitates data handling and analysis. Finally, we present an example program that produces a visualization of the most important data items contained in the files, and can be used as a template script for accessing the provided data. All software discussed below is provided in the code subfolder of the provided datasets, and links to the code repositories are listed in [ssec:code_avail].
% 
% As outlined above, the datasets are stored in two types of files. The primary data, and the spike sorted data, are provided in the data format (in particular, the nev, ns5 and ns6 format) specified by Blackrock Microsystems, the manufacturer of the recording hardware. Second, metadata are provided as one file in the odML format [#Grewe2011]. While data and metadata are provided in documented file formats (see http://blackrockmicro.com/ and http://www.g-node.org/projects/odml, respectively), the mere knowledge of the highly complex internal structure of the files is insufficient to practically make use of their content. In particular, implementations of corresponding loading routines performed from scratch by individual researchers are likely to be incoherent and error-prone. Thus, in the following we will use two community supported open-source libraries to separately load primary data and metadata into a generic, well-defined data representation. 
% 
% We chose the data object model provided by the open-source Neo library [#Garcia2014] as the primary representation of the datasets. Neo provides a hierarchical data structure composed of Python objects that aim to represent electrophysiological data in a generic manner. In addition, Neo provides a number I/Os that enable the user to read from (and in part, write to) a large number of open and vendor-specific file formats. In particular, Neo provides an I/O module for the file format used by Blackrock Microsystems (class BlackrockIO in file neo.io.blackrockio.py). The output of this I/O is a Neo data structure that is a faithful representation of the contents of the primary data files. For detailed information on the structure of the Neo data object model, please consult [#Garcia2014] and the online documentation (http://neo.readthedocs.io/en/latest/index.html). The specific realization of the Neo output structure generated by the readblock() and readsegment() methods of the BlackrockIO are given in detail in the I/O's documentation. 
% 
% Here, we briefly summarize the output of the reach-to-grasp datasets obtained when calling the I/O. The readblock() method of an instantiation of the BlackrockIO returns a Neo Block object as a top level grouping object representing one recording session. In the hierarchy directly below the Block is one single Segment object spanning the complete continuous recording, and one ChannelIndex object for each of the 96 electrodes of the Utah Array ([ssec:surgery_arrayloc]) and each of the 6 sensor signals monitoring the target object manipulation ([subsec:Experimental-Apparatus]). The data from these 102 recording channels is each saved in one AnalogSignal object. All of these are linked to the Segment and the respective ChannelIndex object. Likewise, the spike times (and optionally, the spike waveforms) of each identified unit are saved to a SpikeTrain object. As for the AnalogSignal objects, these are linked to the Segment, and to the ChannelIndex object via a Unit object. Finally, all digital events are saved into a single Event object that lists their time of occurrences and the corresponding event IDs. Additional information from the file is provided as annotations on each individual Neo object (accessible via the annotation property of the object), in particular as annotations to the top level Block object. Note, that although this generic I/O can be used to access the raw data records, no interpretation of the file contents is given. For example, digital events are not interpreted as behavioral events, but only given as the raw numeric codes shown in [Fig:Setup_Overview]. 
% 
% In order to access the metadata stored in the odML file, we use the corresponding library API python-odML described in [#Grewe2011]. In short, odML files store metadata in form of hierarchically structured key-value pairs. The odML files accompanying the provided datasets contain extensive metadata grouped into the following eight top-level sections: Project (general information on the reach-to-grasp project), Subject (information on the monkey), Setup (details of the experimental apparatus), Cerebus (settings of the recording system), UtahArray (information on the multi-electrode array including spike sorting results and the corresponding quality assessement), Headstage (general settings), Recording (task settings, trial definitions with event time stamps and periods), PreProcessing (results of LFP quality assessment and general information on the spike sorting procedure). A tutorial on how to work with the odML library can be found in the online documentation shipped with the library (https://g-node.github.io/python-odml/), and a more detailed description of how to manage metadata by example of the odML framework can be found in [#Zehl2016]. In short, the library supports to read the content of an odML file, provides an API to navigate through the hierarchical structure, and to extract metadata of interest from the key-value pairs. Thus, the python-odML library provides a standardized way to access stored metadata records. 
% 
% As a next step, we combine the primary data and metadata in a manner that is specific to this experiment and aids the analysis process. To this end, the relevant metadata that were extracted from the odML are attached as annotations to data objects in the hierarchical Neo structure. For example, metadata information for a particular single unit originating from the spike sorting process may be attached to the Neo objects representing the sorted spike data of that unit. The task of combining the primary data and metadata is performed by a custom-written Python class named ReachGraspIO that is derived as child class from Neo's BlackrockIO class. For a full documentation of the input arguments, methods, and outputs of this class, please refer to the class documentation in reachgraspio.py. In short, invoking the readblock() method of the ReachGraspIO performs the following steps under the hood: (i) read the primary data using the readblock() method of the parent class (BlackrockIO) as described above, (ii) read the metadata using the python-odML library, (iii) interpret event data based on the digital events (e.g., detect trial start or reward), and (iv) add relevant metadata to the Neo data object using the annotation mechanism. Thus, the Neo Block object returned by the ReachGraspIO contains extensive information attached as annotations of the individual Neo objects, in particular, about whether a SpikeTrain is classified as SUA or MUA, about the spatial positioning of electrodes, or about the identities of electrodes that should be rejected. A full list of these metadata annotations can be found in the documentation of the readblock() method in the file reachgraspio.py.
% 
% In summary, for practical purposes, the resulting data structure of the ReachGraspIO hosts a complete representation of the data and a synthesis of the metadata relevant for analysis. This representation may be saved to disk in a standardized container format (e.g., .mat or HDF5), such that the exact same data and metadata context can also be accessed from other programming languages. For illustration, we provide the data object in the Matlab file format (.mat) in the folder datasets_matlab, containing Matlab structs resembling the Python Neo objects.
% 
% In the following we demonstrate how to use the ReachGraspIO in practice in order to load and visualize the datasets. We follow the file example.py, which is contained as part of the code included with the published datasets. The goal of this program is to create a figure showing the raw signal, LFP, spikes (time stamps and waveforms), and events in a time window (referred to as analysis epoch) around TS-ON of trial 1 for electrode ID 62. 
% 
% In a first step, we load the data using the ReachGraspIO. Considering that only for monkey N an online filtered version of the LFP data is available in the ns2 file, in the following we calculate offline an LFP signal from all raw signals contained in the ns5 or ns6 files using a non-causal low-pass Butterworth filter implemented in the Electrophysiology Analysis Toolkit (Elephant; http://neuralensemble.org/elephant/), which provides analysis capabilities for data stored in the Neo representation. The parameters of this filter are chosen identical to those of the causal filter for the LFP recorded online in monkey N ([subsec:Neural-recording-platform]).
% 
% In a subsequent step, we extract all TS-ON events in correctly performed trials. To this end, we use the function get_events() contained in the utility module neo_utils.py. The function extracts a list of events contained in one Event object of the loaded Neo Block given the filter criteria specified by the parameter event_prop. In our example, the used filter criteria select all events from the Event object “TrialEvents” with a trial_event_labels annotation set to TS-ON, and a performance_in_trial annotation indicating a correct trial.
% 
% In a next step, we create Epoch objects representing analysis epochs around the extracted TS-ON events. To this end, we use add_epochs() also contained in the utility module neo_utils.py. The function excepts the previously extracted TS-ON events as trigger, and defines epochs of a given duration around this trigger. The resulting Epoch object is called “analysis_epochs”. 
% 
% Next, we cut the data according to the analysis epochs and align the cutouts in time. This operation is performed by cut_segment_by_epoch, which returns a list of Segment objects, each containing data of one analysis epoch. The Segments are annotated by the corresponding annotations of the Neo Epoch. In addition, the list of Segment objects is grouped in a new Neo Block, named “data_cut_to_analysis_epochs”. This representation now enables the analysis of the data across trials in the defined analysis epochs. 
% 
% In our example, we show how to create a plot of the data of the analysis epoch in one behavioral trial on the selected electrode. To select the Neo Segment corresponding to the first correct behavioral trial from the Block of the cut data obtained in the previous step, we apply the Neo filter() function. 
% 
% From the selected Segment, LFP data and raw signals can be obtained via the AnalogSignal objects referenced by the analogsignals property, while spike trains and corresponding unit waveforms can be extracted from the SpikeTrain objects referenced by the spiketrains property. The remainder of example.py uses the matplotlib library to create a figure of the data.
% 
% All data and metadata files as well as the code described above can be found in the data repository at XXX INSERT LOCATION HERE XXX. The subdirectory datasets contains all data files and the metadata odML-file for the two provided recording sessions. The subdirectory code contains the files example.py and neo_utils.py. For further reference and inspiration this subdirectory also contains the Python scripts generating the data figures of this manuscript. Furthermore, the subdirectories to code contain frozen versions of the required libraries (Neo, python-odml) as well as the custom loading routine combining data and metadata (reachgraspio.py). Finally, the datasets_matlab directory contains the annotated Neo data object containing all primary data saved in the mat-file format. Updated versions of the code, e.g., adapted to new releases of the Neo library, can be found at https://github.com/INM-6/reach-to-grasp-data-publication.
}




\subsection{Discussion}
\todo{
shortcomings of the presented workflow
- moderate reusability (due to template modularization, but not enrichment code)

general requirements for scientific workflow concepts:
- versioning
- reproducibility
- 'out of the box' usability (user friendliness)
- file size issues (with versioning (git/git-annex/gin \& loading of data)
- central data hosting (especially for collaborative work)}
