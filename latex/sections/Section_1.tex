\clearpage
\section{Introduction}
\label{sec:intro}

Performing experiments is a key component of human development and has been the foundation of knowledge gain during the evolution of mankind. However, the exchange of such experimental results requires the long term documentation of the  experimental procedure. This is only possible with the means of writing down the experimental purpose, execution and result since the invention of scripture. Nowadays, tideous manual scripture has largely been replaced by digital information, making information easier transportable, searchable and duplicatable. Therefore all scientific research nowadays relies mainly on digital information acquisition and storage. Although data can be easily stored and transferred with modern technologie, interpretation of research data is not straight forward as datasets are highly diverse between scientific areas. Depending on the area of research, the diversity within a field highly depends on the field, e.g. in fields which require large experimental setups (particle physics / high field FMRI) data formats and structures  there are only a few data formats defined by the community / company producing the corresponding setup. In other fields the diversity of data is bigger since the scientific methods and aims require a diversity of approaches. Unification would require organizational large-scale efforts arcoss the community and implies additional overhead on the level of each experiment.

The diversity in data modalities and file formats promotes also a heterogeneity in data analysis steps and tools used for extraction of scientific findings.

\todo{TODO: go on here with describing need to standardize analyses and the requirement of complete / consistent metadata}
\todo{Introduce: Reproducible vs replicable vs ...}



In \cref{sec:data} we describe published dataset of a complex, collaborative electrophysiological experiment including an extensive metadata collection. We describe the process of data and metadata preparation required for the data publication and discuss the workflow used in this publication to identify strength and weaknesses of the presented approach. In \cref{sec:metadata} we present odMLtables, a tool for facilitation of metadata collection compatible with the previously presented workflow. We demonstrate the embedding of odMLtables in a real-world metadata workflow and highlight the latest features and developments of the tool. \cref{sec:neo} complements the the previous section by introducing tools for standardized data representation and presents three example applications. \cref{sec:workflows} introduces modern workflow management software for efficient organization and structuring of scientific projects. Finally, all previously presented tools are integrated in a systematic fashion in \cref{sec:implementation} using modern workflow management software to coordinate the application of data and metadata software in a neuroscientific project.
